%#################### 8.1 ####################
\subsection{\hll{Words highlighting \xmybox{1}}}
 
\begin{tabular}{c | c}
\begin{minipage}[m]{0.4\textwidth}
\enum{The \mybox[green]{quick} brown \mybox{fox} \mybox[blue]{jumps} over the
\mybox[green]{lazy} \mybox{dog}.\par
The \xmybox[green]{quick} brown \xmybox{fox} \xmybox[blue]{jumps} over the
\xmybox[green]{lazy} \xmybox{dog}.}{\thesubsection}

\end{minipage}
&
\begin{minipage}[m]{0.55\textwidth}
\renewcommand\textminus{\mbox{-}}%<<<<<<<<<<<
\begin{lstlisting}[numberstyle=\zebra{green!15}{yellow!15},numbers=left,basicstyle=\ttfamily\footnotesize]{tex}
\documentclass{article}
\usepackage{tcolorbox}
\newtcbox{\mybox}[1][red]{on line,
arc=0pt,outer arc=0pt,colback=#1!10!white,colframe=#1!50!black,
boxsep=0pt,left=1pt,right=1pt,top=2pt,bottom=2pt,
boxrule=0pt,bottomrule=1pt,toprule=1pt}
\newtcbox{\xmybox}[1][red]{on line,
arc=7pt,colback=#1!10!white,colframe=#1!50!black,
before upper={\rule[-3pt]{0pt}{10pt}},boxrule=1pt,
boxsep=0pt,left=6pt,right=6pt,top=2pt,bottom=2pt}
\begin{document}
The \mybox[green]{quick} brown \mybox{fox}...\par
The \xmybox[green]{quick} brown \xmybox{fox} ...
\end{document}
\end{lstlisting}
\end{minipage}
\end{tabular}
 
%#################### 8.2 ####################
\subsection{\hll{Unusual words highlighting}}
 
\begin{tabular}{c | c}
\begin{minipage}[m]{0.4\textwidth}
\enum{
Here You can see \mylib{\href{https://texdoc.org/serve/tcolorbox.pdf/0}{more examples}} and learn something new.}{\thesubsection}
\end{minipage}
&
\begin{minipage}[m]{0.55\textwidth}
\renewcommand\textminus{\mbox{-}}%<<<<<<<<<<<
\begin{lstlisting}[numberstyle=\zebra{green!15}{yellow!15},numbers=left,basicstyle=\ttfamily\scriptsize]{tex}
\usepackage[many]{tcolorbox}
\newtcbox{\mylib}{enhanced,nobeforeafter, tcbox raise base, boxrule=0.4pt, top=0mm, bottom=0mm,
  right=0mm, left=4mm, arc=1pt, boxsep=2pt, before upper={\vphantom{dlg}},  colframe=green!50!black, coltext=green!25!black, colback=green!10!white,  overlay={\begin{tcbclipinterior} \fill[green!75!blue!50!white] (frame.south west) rectangle node[text=white,font=\sffamily\bfseries\tiny,rotate=90] {TYP} ([xshift=4mm]frame.north west);\end{tcbclipinterior}}}
\begin{document}
\mylib{recieve}
\end{document}
\end{lstlisting}
\end{minipage}
\end{tabular}
 
 

%#################### 8.3 ####################
\subsection{\hll{Colored circles}}
 
\begin{tabular}{c | c}
\begin{minipage}[m]{0.4\textwidth}
\enum{
\circled[fill=amber,draw=black]{1} 
\circled[fill=babyblue,draw=black]{2} 
\circled[fill=green,draw=black]{3}  
$\cdots$\circled[fill=green!75!blue!50!white,draw=black]{4} 
\circled[fill=orange,draw=black]{5} 
\circled[fill=purple!70!white,draw=black]{6}}{\thesubsection}
\end{minipage}
&
\begin{minipage}[m]{0.55\textwidth}
\renewcommand\textminus{\mbox{-}}%<<<<<<<<<<<
\begin{lstlisting}[numberstyle=\zebra{green!15}{yellow!15},numbers=left,basicstyle=\ttfamily\scriptsize]{tex}
\usepackage{tikz}
\usepackage[framemethod=TikZ]{mdframed}
\usepackage{xcolor}
\usetikzlibrary{calc}
\makeatletter
\newlength{\mylength}
\xdef\CircleFactor{1.1}
\setlength\mylength{\dimexpr\f@size pt}
\newsavebox{\mybox}
\newcommand*\circled[2][draw=blue]{\savebox\mybox{\vbox{\vphantom{WL1/}#1}}\setlength\mylength{\dimexpr\CircleFactor\dimexpr\ht\mybox+\dp\mybox\relax\relax}\tikzset{mystyle/.style={circle,#1,minimum height={\mylength}}}	\tikz[baseline=(char.base)]
\node[mystyle] (char) {#2};}
\makeatother
\definecolor{amber}{rgb}{1.0, 0.75, 0.0}
\definecolor{babyblue}{rgb}{0.54, 0.81, 0.94}
usage -->  \circled[fill=amber,draw=black]{1} 
\end{lstlisting}
\end{minipage}
\end{tabular}
 

%#################### 8.4 ####################
\subsection{\hll{Whole line colored}}
 
\begin{tabular}{c | c}
\begin{minipage}[m]{0.4\textwidth}
\enum{
\hly{green}{some text}
\hly{yellow}{some text}
\hly{red}{some text}}{\thesubsection}
\end{minipage}
&
\begin{minipage}[m]{0.55\textwidth}
\renewcommand\textminus{\mbox{-}}%<<<<<<<<<<<
\begin{lstlisting}[numberstyle=\zebra{green!15}{yellow!15},numbers=left,basicstyle=\ttfamily\scriptsize]{tex}
\documentclass{article}
\usepackage{xcolor}
\newcommand{\hly}[2]{\colorbox{#1!80}{\parbox{\textwidth}{#2}}}

\begin{document}
%\hly{YOURcolor}{some text}
\hly{green}{some text}
\hly{yellow}{some text}
\hly{red}{some text}
\end{document}
\end{lstlisting}
\end{minipage}
\end{tabular}
 

%#################### 8.5 ####################
\subsection{\hll{Circle text in points to other text}}
 
\begin{tabular}{c | c}
\begin{minipage}[m]{0.4\textwidth}
\enum{\tikzset{mynode/.style={inner sep=2pt,fill=cyan!50,draw=blue,line width=1pt,rounded corners}}

\tikzmarknode[mynode]{A}{This} is just some text that I will repeat for this section again and again. This is just some text that I will repeat for this section again and again. 

\begin{tikzpicture}[remember picture, overlay]
    \draw[->,line width=1pt,blue] (A) --++ (1,1) node[above right] {your comment here};
\end{tikzpicture}}{\thesubsection}
\end{minipage}
&
\begin{minipage}[m]{0.55\textwidth}
\renewcommand\textminus{\mbox{-}}%<<<<<<<<<<<
\begin{lstlisting}[numberstyle=\zebra{green!15}{yellow!15},numbers=left,basicstyle=\ttfamily\scriptsize]{tex}
\documentclass{article}
\usepackage{tikz}
\usetikzlibrary{tikzmark}

\begin{document}
\tikzset{mynode/.style={inner sep=2pt,fill=cyan!50,draw=blue,line width=1pt,rounded corners}}

This is just some \tikzmarknode[mynode]{A}{text that} I will repeat for this section again and again. This is just some text that I will repeat for this section again and again. 

\begin{tikzpicture}[remember picture, overlay]
    \draw[->,line width=1pt,blue] (A) --++ (1,1) node[above right] {your comment here};
\end{tikzpicture}

\end{document}
\end{lstlisting}
\end{minipage}
\end{tabular}
 

%#################### 8.6 ####################
\subsection{\hll{Keybutton}}
 
\begin{tabular}{c | c}
\begin{minipage}[m]{0.4\textwidth}
\enum{Press \button{alt } + \button{F4 } for help !}{\thesubsection}
\end{minipage}
&
\begin{minipage}[m]{0.55\textwidth}
\renewcommand\textminus{\mbox{-}}%<<<<<<<<<<<
\begin{lstlisting}[numberstyle=\zebra{green!15}{yellow!15},numbers=left,basicstyle=\ttfamily\scriptsize]{tex}
\documentclass[10pt]{article}
\usepackage{tikz}
\usetikzlibrary{shadows}
\tikzstyle{buttonstyle} = [rectangle, fill = black!30, draw = black!80, drop shadow, font={\sffamily\bfseries}, text=white]
\newcommand*{\button}[1]{\tikz{\node[buttonstyle] {#1};}}

\begin{document}
Press \button{F5} for help !
\end{document}
\end{lstlisting}
\end{minipage}
\end{tabular}
 


%#################### 8.7 ####################
\subsection{\hll{Colorful \xmyboxg{$\backslash$tableofcontents}}}
 
\begin{tabular}{c | c}
\begin{minipage}[m]{0.4\textwidth}
\enum{Press \button{alt } + \button{F4 } for help !}{\thesubsection}
\end{minipage}
&
\begin{minipage}[m]{0.55\textwidth}
\renewcommand\textminus{\mbox{-}}%<<<<<<<<<<<
\begin{lstlisting}[numberstyle=\zebra{green!15}{yellow!15},numbers=left,basicstyle=\ttfamily\scriptsize]{tex}
\documentclass{article}
\usepackage{tocloft}
\usepackage{xcolor}
\usepackage{tikz}
\usetikzlibrary{backgrounds}
\usetikzlibrary{calc}
\newcounter{seccntr}
\setcounter{seccntr}{-1}
\newcommand*{\hnode}[1]{%
\tikz[remember picture] \node[minimum size=0pt,inner sep=0pt,outer sep=4.5pt] (#1) {};}
\renewcommand{\cftsecfont}{\hnode{P1}\bfseries\Large
\stepcounter{seccntr}%
\ifcase\value{seccntr}%
\tikz[remember picture,overlay] \draw (P1.north west)  [line width={17pt}, red,opacity=0.3] -- ++($(\textwidth,0) + (1ex,0)$);
\or\tikz[remember picture,overlay] \draw (P1.north west)  [line width={17pt}, green,opacity=0.4] -- ++($(\textwidth,0) + (1ex,0)$);
\or\tikz[remember picture,overlay]  \draw (P1.north west)  [line width={17pt}, yellow,opacity=1] -- ++($(\textwidth,0) + (1ex,0)$);
\or\tikz[remember picture,overlay]  \draw (P1.north west)  [line width={17pt}, blue,opacity=0.6] -- ++($(\textwidth,0) + (1ex,0)$);
\or\tikz[remember picture,overlay]  \draw (P1.north west)  [line width={17pt}, orange,opacity=0.7] -- ++($(\textwidth,0) + (1ex,0)$);
\else\tikz[remember picture,overlay] \draw (P1.north west)  [line width={17pt}, gray,opacity=0.8] -- ++($(\textwidth,0) + (1ex,0)$);%-- default
\fi  %
}
\renewcommand{\cftsecpagefont}{\bfseries}

\begin{document}
\tableofcontents
\section{First Section}\subsection{\hll{A su
bsubsection}}\subsection{\hll{A su
bsubsection}}
\section{Second Section}\subsection{\hll{A su
bsubsection}}
\section{Third Section}
\end{document}
\end{lstlisting}
\end{minipage}
\end{tabular}
 

%#################### 8.8 ####################
