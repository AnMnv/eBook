%#################### 4.1 ####################
\subsection{\hll{Nice tcolorbox}}
\begin{table}[h!]
\begin{tabular}{c | c}
\begin{minipage}[m]{0.4\textwidth}
\enum{
\begin{tcolorbox}[colback=white!100,colframe=red!75!black,width=7cm,righttitle=0.5cm,subtitle style={boxrule=0.4pt, colback=yellow!50!red!25!white},title= \bf{1}\hfill  \bf{22}]
	\begin{center}\bf{333}\end{center}
	\tcblower
	\href{https://tools.ietf.org/doc/texlive-doc/latex/tcolorbox/tcolorbox.pdf}{Source}
	\end{tcolorbox}}{\thesubsection}
\end{minipage}
&
\begin{minipage}[m]{0.55\textwidth}
\renewcommand\textminus{\mbox{-}}%<<<<<<<<<<<
\begin{lstlisting}[numberstyle=\zebra{green!15}{yellow!15},numbers=left,basicstyle=\scriptsize]{tex}
\PassOptionsToPackage{svgnames}{xcolor}
\documentclass[twocolumn,a4paper]{article}
\usepackage{tcolorbox}
\tcbuselibrary{skins,breakable}
\usetikzlibrary{shadings,shadows}%preambule
\begin{tcolorbox}[colback=white!100,colframe=red!75!black,width=7cm,righttitle=0.5cm, subtitle style={boxrule=0.4pt,colback=yellow!50!red!25!white},title= \bf{1}\hfill \bf{22}]
	\begin{center}\bf{333}\end{center}
	\tcblower
	\href{https://tools.ietf.org/doc/texlive-doc/latex/tcolorbox/tcolorbox.pdf}{URL}
\end{tcolorbox}
\end{lstlisting}
\end{minipage}
\end{tabular}
\end{table} 

%#################### 4.2 ####################
\subsection{\hll{Color box with yellow border}}
\begin{table}[h!]
\begin{tabular}{c | c}
\begin{minipage}[m]{0.4\textwidth}
\enum{\begin{mycolorbox}{Remarque}
Some text inside
\end{mycolorbox}}{\thesubsection}
\end{minipage}
&
\begin{minipage}[m]{0.55\textwidth}
\renewcommand\textminus{\mbox{-}}%<<<<<<<<<<<
\begin{lstlisting}[numberstyle=\zebra{green!15}{yellow!15},numbers=left,basicstyle=\scriptsize]{tex}
\documentclass[border=2mm]{standalone}
\usepackage[most]{tcolorbox}
\usepackage{lipsum}

\newtcolorbox{mycolorbox}[1]{
    enhanced,    breakable,
    title=#1,    colback=white,
    colbacktitle=green!20!white,
    coltitle=black,
    fonttitle=\bfseries,
    boxrule=.5pt,    arc=0pt,
    outer arc=0pt,
    colframe=yellow!80!orange,
    borderline west={2pt}{0pt}{red}   }

\begin{document}
\begin{mycolorbox}{Remarque}
\lipsum[1]
\end{mycolorbox}
\end{document}
\end{lstlisting}
\end{minipage}
\end{tabular}
\end{table}
 

%#################### 4.3 ####################
\subsection{\hll{A drop capital in a tcolorbox}}
\begin{tabular}{c | c}
\begin{minipage}[m]{0.4\textwidth}
\enum{\begin{tcolorbox}
\lettrine{S}{ome} text. \lipsum[1][1]\par
\end{tcolorbox}}{\thesubsection}
\end{minipage}
&
\begin{minipage}[m]{0.55\textwidth}
\renewcommand\textminus{\mbox{-}}%<<<<<<<<<<<
\begin{lstlisting}[numberstyle=\zebra{green!15}{yellow!15},numbers=left,basicstyle=\ttfamily\footnotesize]{tex}
\documentclass{article}
\usepackage{lettrine}
\usepackage{tcolorbox}
\usepackage{lipsum}

\begin{document}
\begin{tcolorbox}
\lettrine{S}{ome} text. \lipsum[1]
\end{tcolorbox}
\end{document}
\end{lstlisting}
\end{minipage}
\end{tabular}
 

%#################### 4.4 ####################
\subsection{\hll{\textit{Table with the desired length. }}}
\begin{table}[h!]
\begin{tabular}{c | c}
\begin{minipage}[m]{0.4\textwidth}
\enum{
\includegraphics[width=1\linewidth]{4.5.png}
\textit{a command was also created to make a new cell view in the table}
}{\thesubsection}
\end{minipage}
&
\begin{minipage}[m]{0.55\textwidth}
\renewcommand\textminus{\mbox{-}}%<<<<<<<<<<<
\begin{lstlisting}[numberstyle=\zebra{green!15}{yellow!15},numbers=left,basicstyle=\ttfamily\scriptsize]{tex}
\usepackage{graphicx}
\usepackage{tabularx}
\newcolumntype{Y}{>{\centering\arraybackslash}X}
\begin{document}
\begin{table}[h!]
\begin{center}
\caption{\textbf{Caption}}
  \begin{tabularx}{14cm}{|Y|Y|c|Y|Y|}
  \hline
  Variant & res & Veriaty of waters $f_0$, res & C, res & L, res\\
  \hline
  5       &     1   &               2               & 1.26 & 5\\
  \hline
  \end{tabularx}
\end{center}
\end{table}
\end{lstlisting}
\end{minipage}
\end{tabular}
\end{table}

%#################### 4.5 ####################
\subsection{\hll{Photo positioning}}
\begin{tabular}{c | c}
\begin{minipage}[m]{0.4\textwidth}
\enum{
\begin{tcolorbox}[enhanced,sharp corners,
width={5cm},
colback=white,
overlay={\node at (frame.south east) {\includegraphics[scale=0.1]{example-image-a}};} ]
Sample text here.
\end{tcolorbox}}{\thesubsection}
\end{minipage}
&
\begin{minipage}[m]{0.55\textwidth}
\renewcommand\textminus{\mbox{-}}%<<<<<<<<<<<
\begin{lstlisting}[numberstyle=\zebra{green!15}{yellow!15},numbers=left,basicstyle=\ttfamily\footnotesize]{tex}
\documentclass{article}
\usepackage[most]{tcolorbox}
\usepackage{graphicx}
\begin{document}
\begin{tcolorbox}[enhanced,sharp corners,
width={5cm},
colback=white,
overlay={\node at (frame.south east) {\includegraphics[scale=0.1]{example-image-a}};} ]
Sample text here.
\end{tcolorbox}
\end{document}  
\end{lstlisting}
\end{minipage}
\end{tabular}

\clearpage

%#################### 4.6 ####################
\subsection{\hll{bclogo – Creating colourful boxes with logos}}
\begin{table}[h!]
\begin{tabular}{c | c}
\begin{minipage}[m]{0.4\textwidth}
\enum{\href{https://ctan.org/pkg/bclogo}{\img{1}{C:/Users/user/Desktop/eBook/images/4.5/4.5.pdf}}}{\thesubsection}
\end{minipage}
&
\begin{minipage}[m]{0.55\textwidth}
\renewcommand\textminus{\mbox{-}}%<<<<<<<<<<<
\begin{lstlisting}[numberstyle=\zebra{green!15}{yellow!15},numbers=left,basicstyle=\ttfamily\scriptsize]
\documentclass{article}
\usepackage{geometry}
\geometry{
paperwidth=8cm,
paperheight=14cm,
margin=0.5cm
}
\usepackage{xcolor}
\usepackage[most]{tcolorbox}
\usepackage[tikz]{bclogo}

\newtcolorbox{framedd}[1][]{
  colframe=lightgray,
  colback=yellow!40!white,
  enhanced jigsaw,
  sharp corners,
  lower separated=false,
  lefthand width=1cm,
  sidebyside gap=0.5cm,
  sidebyside,#1}

\begin{document}
\begin{framedd}
  \bcbombe  \tcblower  Some text inside.
\end{framedd}

\begin{framedd}[colback=blue!40!green]
  \bclampe   \tcblower  Some text inside.
\end{framedd} 

\begin{framedd}
  \bcattention  \bcinterdit  \tcblower
  Some text inside.
\end{framedd}

\begin{framedd}[colback=blue!40!green]
   \bcnucleaire  \tcblower
  Some text inside.
\end{framedd}

\begin{framedd}[colback=blue!40!green]
 \bcdanger \tcblower
  Some text inside.
\end{framedd}

\begin{framedd}
  \bcquestion \tcblower
  Some text inside.
\end{framedd}

\begin{framedd}[colback=blue!40!green, lefthand width=2.5cm]
  \bcsoleil  \bceclaircie \bcpluie  \bcneige \tcblower
  Some text inside.
\end{framedd}

\begin{framedd}[lefthand width=3cm]
  \bccube \bcdodecaedre \bcicosaedre \bcoctaedre \bctetraedre  \tcblower
  Some text inside.
\end{framedd}
\end{document}
\end{lstlisting}
\end{minipage}
\end{tabular}
\end{table}

  
\clearpage
%#################### 4.7 ####################
\subsection{\hll{Warning banner}}
\begin{tabular}{c | c}
\begin{minipage}[m]{0.4\textwidth}
\enum{
\begin{caja}[title=warning]
Here is some text 
\end{caja}}{\thesubsection}
\end{minipage}
&
\begin{minipage}[m]{0.55\textwidth}
\renewcommand\textminus{\mbox{-}}%<<<<<<<<<<<
\begin{lstlisting}[numberstyle=\zebra{green!15}{yellow!15},numbers=left,basicstyle=\ttfamily\footnotesize]{tex}
\usepackage[utf8]{inputenc}
\usepackage[T1]{fontenc}
\usepackage[most]{tcolorbox}
\definecolor{orang}{RGB}{255,155,0}
\newtcolorbox[auto counter,number within=section]{caja}[1][]{
enhanced jigsaw,colback=white,colframe=orang,coltitle=orang,
fonttitle=\bfseries\sffamily,
sharp corners,
detach title,
leftrule=10mm,
% What you need %%%%%%%%%%%%
underlay unbroken and first={\node[below,text=black,anchor=east]
at ([xshift=-5.5pt]interior.base west) {\Huge  \textbf{!}};},
%%%%%%%%%%%%%%%%%%%%%%%%
breakable,pad at break=1mm,
#1,
code={\ifdefempty{\tcbtitletext}{}{\tcbset{before upper={\tcbtitle\par\medskip}}}},}
\begin{document}
\begin{caja}[title=warning]
The vertical alignment settings 
\end{caja}
\end{document}	
\end{lstlisting}
\end{minipage}
\end{tabular}

\vspace{0.2cm}	

%#################### 4.8 ####################
\subsection{\hll{Absolutely centered cells (vertically and horisontally)}}
\begin{tabular}{c | c}
\begin{minipage}[m]{0.4\textwidth}
\enum{
\makegapedcells
    \begin{tabular}{|c|c|c| }
    \hline
all&in&cells\\ \hline
are&centered&vertically\\ \hline
and&horisontally&$\sum$\\ \hline
 
\end{tabular}
 }{\thesubsection}
\end{minipage}
&
\begin{minipage}[m]{0.55\textwidth}
\renewcommand\textminus{\mbox{-}}%<<<<<<<<<<<
\begin{lstlisting}[numberstyle=\zebra{green!15}{yellow!15},numbers=left,basicstyle=\ttfamily\footnotesize]{tex}
\documentclass{article}
\usepackage{float}
\usepackage{array, makecell}
\setcellgapes{5pt}

\begin{document}
\begin{table}[H]
\center
\makegapedcells
    \begin{tabular}{|c|c|c|c|}
    \hline
1&1&1&1\\ \hline
1&1&1&1\\ \hline
1&1&1&1\\ \hline
 
\end{tabular}
\end{table}

\end{document}
\end{lstlisting}
\end{minipage}
\end{tabular}

%#################### 4.9 ####################
\subsection{\hll{Martix made of table}}
\begin{tabular}{c | c}

\begin{minipage}[m]{0.4\textwidth}
\enum{    
 
\begin{tabular}{l|l c r|l}
 
        & $a_{1,1}$ & $\dots, a_{1,n}$ & 0 &                 \\  
        & $a_{1,1}$ & $\dots, a_{1,n}$ & 0 &                 \\ 
        & \multicolumn{3}{l|}{\dotfill}   &                  \\  
        & $a_{1,1}$ & $\dots, a_{1,n}$ & 0 &                 \\ 
$d_{n+1}$ &         &                &   & = 0 \\  
        & $a_{1,1}$ & $\dots, a_{1,n}$ & 0 &                 \\  
        & $a_{1,1}$ & $\dots, a_{1,n}$ & 0 &                 \\ 
        & \multicolumn{3}{l|}{\dotfill}   &                  \\ 
        & $a_{1,1}$ & $\dots, a_{1,n}$ & 0 &                 \\ 
\end{tabular}
 }{\thesubsection}

\end{minipage}
&
\begin{minipage}[m]{0.55\textwidth}
\renewcommand\textminus{\mbox{-}}%<<<<<<<<<<<
\begin{lstlisting}[numberstyle=\zebra{green!15}{yellow!15},numbers=left,basicstyle=\ttfamily\footnotesize]{tex}
\documentclass[a4paper,14pt]{extreport}
\begin{document}
\begin{table}[]
\begin{tabular}{l|l c r|l}
& $a_{1,1}$ & $\dots, a_{1,n}$ & 0 &                 \\  
& $a_{1,1}$ & $\dots, a_{1,n}$ & 0 &                 \\ 
& \multicolumn{3}{l|}{\dotfill}   &                  \\  
& $a_{1,1}$ & $\dots, a_{1,n}$ & 0 &                 \\ 
$d_{n+1}$ &         &    &   & = $\pm 2ad_n$ = 0     \\  
& $a_{1,1}$ & $\dots, a_{1,n}$ & 0 &                 \\  
& $a_{1,1}$ & $\dots, a_{1,n}$ & 0 &                 \\ 
& \multicolumn{3}{l|}{\dotfill}   &                  \\ 
& $a_{1,1}$ & $\dots, a_{1,n}$ & 0 &                 \\ 
\end{tabular}
\end{table}
\end{document}
\end{lstlisting}
\end{minipage}
\end{tabular}

%#################### 4.10 ####################
\subsection{\hll{Centering cells with \xmybox{NiceTabular}}}
\begin{tabular}{c | c}
\begin{minipage}[m]{0.4\textwidth}
\enum{ \begin{NiceTabular}{|c|c|c|} 
\hline
\cellcolor{red}1& \cellcolor{green}1 & \cellcolor{black!10}EVERY\\\hline 
\cellcolor{orange}1 & \cellcolor{red!35}1 & \cellcolor{brown!50}CELL \\ \hline
\cellcolor{green!35}1 & \cellcolor{blue!45}1 & \cellcolor{yellow}CENTERED \\ \hline
\end{NiceTabular}  }{\thesubsection}
\end{minipage}
&
\begin{minipage}[m]{0.55\textwidth}
\renewcommand\textminus{\mbox{-}}%<<<<<<<<<<<
\begin{lstlisting}[numberstyle=\zebra{green!15}{yellow!15},numbers=left,basicstyle=\ttfamily\footnotesize] 
\documentclass{article}
\usepackage[table]{xcolor}
\usepackage{nicematrix}
\NiceMatrixOptions{cell-space-top-limit=5pt,cell-space-bottom-limit=5pt}

\begin{document}
\begin{table}[htbp]
\centering
\begin{NiceTabular}{|c|c|c|} 
\hline
\cellcolor{red}1& \cellcolor{green}1 &  1 \\ \hline 
\cellcolor{orange}1 & \cellcolor{red!35}1 &  1 \\ \hline
\cellcolor{green!35}1 & \cellcolor{blue!45}1 &  1 \\ \hline
\end{NiceTabular}
\end{table}
\end{document}
\end{lstlisting}
\end{minipage}
\end{tabular}


%#################### 4.11 ####################
\subsection{\hll{Centered cells in \xmybox{longtable}}}
\begin{tabular}{c | c}
\begin{minipage}[m]{0.4\textwidth}
\enum{\includegraphics[width=1.\linewidth]{4.11.png}}{\thesubsection}
\end{minipage}
&
\begin{minipage}[m]{0.55\textwidth}
\renewcommand\textminus{\mbox{-}}%<<<<<<<<<<<
\begin{lstlisting}[numberstyle=\zebra{green!15}{yellow!15},numbers=left,basicstyle=\ttfamily\scriptsize] 
\documentclass{article}
\usepackage[left=1.5cm,right=1.5cm,
top=1.5cm,bottom=2cm,bindingoffset=0cm]{geometry}
\usepackage{float}
\usepackage{array, makecell}
\usepackage[utf8]{inputenc}
\usepackage{lipsum}
\usepackage{booktabs}
\usepackage{multirow}
\usepackage{pdflscape}
\usepackage{longtable, array}

\begin{document}
\begin{landscape}
\begin{longtable}{@{} *{2}{m{.15\paperwidth}} *{1}{m{.40\paperwidth}} @{}}
\endfirsthead
\endhead
\toprule
\textbf{Enum} & \textbf{Example} & \textbf{Description} \\
\midrule
1 & test & \lipsum[50]\\
\midrule
2a & test & \lipsum[50]\\
2b & test & \lipsum[50]\\
\bottomrule
\end{longtable}
\end{landscape}
\end{document}          
\end{lstlisting}
\end{minipage}
\end{tabular}

%#################### 4.12 ####################
\subsection{\hll{If table is not wide enough}}
\begin{tabular}{c | c}
\begin{minipage}[m]{0.4\textwidth}
\enum{ 
\begin{tabularx}{\textwidth}{X  X  X  X}
       & Item1 & Item2 & Item3 \\ \midrule
Group1 & 0.8   & 0.1   & 0.1  \\
Group2 & 0.1   & 0.8   & 0.1  \\
Group3 & 0.1   & 0.1   & 0.8  \\
Group4 & 0.34  & 0.33  & 0.33 \\ \bottomrule
\end{tabularx}}{\thesubsection}
\end{minipage}
&
\begin{minipage}[m]{0.55\textwidth}
\renewcommand\textminus{\mbox{-}}%<<<<<<<<<<<
\begin{lstlisting}[numberstyle=\zebra{green!15}{yellow!15},numbers=left,basicstyle=\ttfamily\footnotesize] 
\documentclass{article}
\usepackage[left=1.5cm,right=1.5cm,
top=1.5cm,bottom=2cm,bindingoffset=0cm]{geometry}
\usepackage{graphicx}
\usepackage{booktabs}
\usepackage{tabularx}

\begin{document}         
\begin{table}[!ht] 
\caption{Vertical and lateral stresses of mortar.}  
\vspace{0.5cm}
\begin{tabularx}{\textwidth}{X  X  X  X}
       & Item1 & Item2 & Item3 \\ \midrule
Group1 & 0.8   & 0.1   & 0.1  \\
Group2 & 0.1   & 0.8   & 0.1  \\
Group3 & 0.1   & 0.1   & 0.8  \\
Group4 & 0.34  & 0.33  & 0.33 \\ \bottomrule
\end{tabularx}
\label{c}
\end{table}
\end{document}        
\end{lstlisting}
\end{minipage}
\end{tabular}


%#################### 4.13 ####################
\subsection{\hll{Text next to a table}}
\begin{tabular}{c | c}
\begin{minipage}[m]{0.4\textwidth}
\enum{\begin{minipage}[m]{0.4\textwidth}
text  text text
\end{minipage}
\hfill
\begin{minipage}[m]{0.5\textwidth}
\begin{tabular}{|c|c|c|}
\hline
1 & 22 & 333  \\ \hline
  &    &      \\ \hline
  &    &      \\ \hline
  &    &      \\ \hline
\end{tabular}
\end{minipage}}{\thesubsection}
\end{minipage}
&
\begin{minipage}[m]{0.55\textwidth}
\renewcommand\textminus{\mbox{-}}%<<<<<<<<<<<
\begin{lstlisting}[numberstyle=\zebra{green!15}{yellow!15},numbers=left,basicstyle=\ttfamily\footnotesize] 
\documentclass[a4paper,14pt]{extreport}
\usepackage[left=1.5cm,right=1.5cm,top=1.5cm,bottom=2cm,bindingoffset=0cm]{geometry}
\usepackage{lipsum}

\begin{document}
\begin{minipage}[m]{0.58\textwidth}
text text text
\end{minipage}
\hspace{0.2cm}
\begin{minipage}[m]{0.40\textwidth}
\begin{tabular}{|c|c|c|}
\hline
1 & 22 & 333 &  \\ \hline
  &    &     &  \\ \hline
  &    &     &  \\ \hline
  &    &     &  \\ \hline
\end{tabular}
\end{minipage}
\end{document}
\end{lstlisting}
\end{minipage}
\end{tabular}

%#################### 4.14 ####################
\subsection{\hll{Hand Drawn tcolorbox}}
\begin{tabular}{c | c}
\begin{minipage}[m]{0.4\textwidth}
\enum{\begin{mytheo}{}{theoexample}
some text
\end{mytheo}}{\thesubsection}
\end{minipage}
&
\begin{minipage}[m]{0.55\textwidth}
\renewcommand\textminus{\mbox{-}}%<<<<<<<<<<<
\begin{lstlisting}[numberstyle=\zebra{green!15}{yellow!15},numbers=left,basicstyle=\ttfamily\footnotesize] 
\documentclass{article}
\usepackage[most]{tcolorbox}
\usepackage{emerald}
\usetikzlibrary{decorations.pathmorphing}
\usetikzlibrary{shadows}
\tikzset{decoration={random steps,segment length=2mm,amplitude=0.6pt}}
\newtcbtheorem{mytheo}{Theorem}{
  coltitle=green!80!black,
  colback=lightgray!20,
  colbacktitle=lightgray!20,
  fonttitle=\bfseries\ECFAugie,
  enhanced,
  attach boxed title to top left={yshift=-0.18cm,xshift=-0.5mm},
  boxed title style={
    tikz={rotate=4,transform shape},
    frame code={
      \draw[decorate,fill=lightgray!20] (frame.south west) rectangle (frame.north east);
    }  },
  frame code={
    \draw[decorate,fill=lightgray!20,drop shadow] (frame.north east) rectangle (frame.south west);
  },}{th}

\begin{document}
\begin{mytheo}{}{theoexample}
content...
\end{mytheo}
\end{document}
\end{lstlisting}
\end{minipage}
\end{tabular}



%#################### 4.15 ####################
\subsection{\hll{Text next to a table}}
\begin{tabular}{c | c}
\begin{minipage}[m]{0.4\textwidth}
\enum{  \center \begin{tikzpicture}[
    start chain=going below,
    node distance=2mm,
    Node/.style = {minimum width=#1,
                   shape=rectangle, 
                   draw, fill=white,
                   on chain},
    Pattern/.style = {pattern=north east hatch,
                    pattern color=teal!30,
                    hatch distance=7pt, 
                    hatch thickness=2pt},
    font=\small\sffamily]
%----------------
    \node[Node=24mm, Pattern, 
            preaction={fill=white}] (a) {without shadow};
    \begin{scope}[on background layer]
        \node[fit=(a),fill=red] {};
    \end{scope}

    \node[Node=24mm, drop shadow,
            preaction={fill=yellow}, Pattern] (b) {with shadow};

    \node[Node=24mm, preaction={fill=yellow},
            drop shadow, Pattern] (b) {with shadow};

    \node[Node=24mm, postaction={Pattern},
            drop shadow] (b) {with shadow};

    \node[Node=24mm, postaction={draw=red, Pattern},
            drop shadow] (b) {with shadow};

    \node[Node=24mm, drop shadow] (c) {without pattern};

   
%---
 \end{tikzpicture} 
  \href{https://tex.stackexchange.com/questions/154842/using-pattern-inside-tikz-shapes-with-dropped-shadows}{*****} }{\thesubsection}
\end{minipage}
&
\begin{minipage}[m]{0.55\textwidth}
\renewcommand\textminus{\mbox{-}}%<<<<<<<<<<<
\begin{lstlisting}[numberstyle=\zebra{green!15}{yellow!15},numbers=left,basicstyle=\ttfamily\scriptsize] 
\documentclass[tikz,border=5mm]{standalone}
\usetikzlibrary{chains,patterns,shadows,fit,backgrounds}

\makeatletter
\tikzset{% customization of pattern
         % based on <m.wibrow@gm...> - 2013-03-24 07:20: 
        hatch distance/.store in=\hatchdistance,
        hatch distance=5pt,
        hatch thickness/.store in=\hatchthickness,
        hatch thickness=5pt
        }
\pgfdeclarepatternformonly[\hatchdistance,\hatchthickness]{north east hatch}% name
    {\pgfqpoint{-1pt}{-1pt}}% below left
    {\pgfqpoint{\hatchdistance}{\hatchdistance}}% above right
    {\pgfpoint{\hatchdistance-1pt}{\hatchdistance-1pt}}%
    {
        \pgfsetcolor{\tikz@pattern@color}
        \pgfsetlinewidth{\hatchthickness}
        \pgfpathmoveto{\pgfqpoint{0pt}{0pt}}
        \pgfpathlineto{\pgfqpoint{\hatchdistance}{\hatchdistance}}
        \pgfusepath{stroke}
    }
\makeatother

\begin{document}
 \begin{tikzpicture}[
    start chain=going below,
    node distance=2mm,
    Node/.style = {minimum width=#1,
                   shape=rectangle, 
                   draw, fill=white,
                   on chain},
    Pattern/.style = {pattern=north east hatch,
                    pattern color=teal!30,
                    hatch distance=7pt, 
                    hatch thickness=2pt},
    font=\small\sffamily]
%----------------
    \node[Node=24mm, Pattern, 
            preaction={fill=white}] (a) {without shadow};
    \begin{scope}[on background layer]
        \node[fit=(a),fill=red] {};
    \end{scope}

    \node[Node=24mm, drop shadow,
            preaction={fill=yellow}, Pattern] (b) {with shadow};

    \node[Node=24mm, preaction={fill=yellow},
            drop shadow, Pattern] (b) {with shadow};

    \node[Node=24mm, postaction={Pattern},
            drop shadow] (b) {with shadow};

    \node[Node=24mm, postaction={draw=red, Pattern},
            drop shadow] (b) {with shadow};

    \node[Node=24mm, drop shadow] (c) {without pattern};
%---
 \end{tikzpicture}   
\end{document}
\end{lstlisting}
\end{minipage}
\end{tabular}






%#################### 4.16 ####################%https://tex.stackexchange.com/questions/624558/how-can-i-do-this-block-in-beamer
\subsection{\hll{Halfframed boxes}}
\begin{tabular}{c | c}
\begin{minipage}[m]{0.4\textwidth}
\enum{\includegraphics[width=1\linewidth]{4.16.png}}{\thesubsection}
\end{minipage}
&
\begin{minipage}[m]{0.55\textwidth}
\renewcommand\textminus{\mbox{-}}%<<<<<<<<<<<
\begin{lstlisting}[numberstyle=\zebra{green!15}{yellow!15},numbers=left,basicstyle=\ttfamily\footnotesize] 
\documentclass{beamer}
\usepackage[english]{babel}
\usepackage[T1]{fontenc}
\usepackage[utf8]{inputenc}
\usepackage{tikz}
\usepackage{tcolorbox}
\usetikzlibrary{calc}
\tcbuselibrary{skins,breakable,raster}
\makeatletter
\definecolor{myred}{RGB}{209,23,23}
\definecolor{myorange}{RGB}{255,153,51}
\definecolor{mypurple}{RGB}{102,0,102}
\definecolor{mygrey}{RGB}{200,200,200}

\newtcolorbox{mybox}[2]{empty,coltitle = #1,title = #2,overlay ={\draw[mygrey,line width=1pt](frame.north west)--(frame.north east)--(frame.south east)--(frame.south west)--(frame.north west);
\draw[#1,line width=1pt]
($(frame.north west)!0.33!(frame.south west)$)
--(frame.north west)
--($(frame.north west)!0.33!(frame.north east)$);
\draw[#1,line width=1pt]
($(frame.south east)!0.33!(frame.south west)$)
--(frame.south east)
--($(frame.south east)!0.33!(frame.north east)$);}}

\tcbset{marktext/.style={overlay={\node[rotate=90,text=black,anchor=north east] at (frame.north west){#1};}, code={\setbox\z@=\color@hbox#1\color@endbox\tcbdimto\myheight{\wd\z@+3mm}},minimum for equal height group=\tcb@ehgid:\myheight,  }}
\makeatother

\begin{document}
\begin{frame}
\begin{tcbraster}[raster columns=3, raster equal height=rows]
\begin{mybox}{myred}{Title 1}
some text in the first box
\end{mybox}
\begin{mybox}{myorange}{Title 2}
some text in the second box
\end{mybox}
\begin{mybox}{mypurple}{Title 3}
some text in the third box blabla
\end{mybox}
\end{tcbraster}
\end{frame}
\end{document}
\end{lstlisting}
\end{minipage}
\end{tabular}
%#################### 4.17 ####################
\subsection{\hll{Vertically and horizontally align image inside table}}
\begin{tabular}{c | c}
\begin{minipage}[m]{0.4\textwidth}
  \begin{tabular}{|p{1.5cm}|c|}
    \hline
    \textbf{Num.}   
    &   \textbf{Images} \\ 
    \hline
    Nr. 1   &   \includegraphics[width=0.6\textwidth,margin=0pt 1ex 0pt 1ex,valign=m]{example-image}  \\
    \hline
    Nr. 2   &   \includegraphics[width=0.6\textwidth,margin=0pt 1ex 0pt 1ex,valign=m]{example-grid-100x100pt}  \\
    \hline
    \end{tabular}
\end{minipage}
&
\begin{minipage}[m]{0.55\textwidth}
\renewcommand\textminus{\mbox{-}}%<<<<<<<<<<<
\begin{lstlisting}[numberstyle=\zebra{green!15}{yellow!15},numbers=left,basicstyle=\ttfamily\footnotesize] 
\documentclass{article}
\usepackage{graphicx}
\usepackage[export]{adjustbox}

\begin{document}
\begin{table}[htbp]
\centering
\caption{My caption}
\label{tab:mytab}
\begin{tabular}{|p{1.5cm}|c|}
\hline
\textbf{Num.}   
&   \textbf{Images} \\ 
\hline
Nr. 1   &   \includegraphics[width=0.8\textwidth,margin=0pt 1ex 0pt 1ex,valign=m]{example-image}  \\
\hline
Nr. 2   &   \includegraphics[width=0.8\textwidth,margin=0pt 1ex 0pt 1ex,valign=m]{example-grid-100x100pt}  \\
\hline
\end{tabular}
\end{table}
\end{document}
\end{lstlisting}
\end{minipage}
\end{tabular}

%#################### 4.18 ####################
\subsection{\hll{Box with \gradientRGB{nice gradient text}{72,107,234}{60,214,112}}}
\begin{tabular}{c | c}
\begin{minipage}[m]{0.4\textwidth}
Normal text Normal text\\

\href{https://github.com/readme/featured/nasa-ingenuity-helicopter}{\enum{\cooltext{We wanted to make sure everyone was recognized for their 
contributions to this incredible human achievement.}}{\thesubsection}}

Normal text Normal text
\end{minipage}
&
\begin{minipage}[m]{0.55\textwidth}
\renewcommand\textminus{\mbox{-}}%<<<<<<<<<<<
\begin{lstlisting}[numberstyle=\zebra{green!15}{yellow!15},numbers=left,basicstyle=\ttfamily\footnotesize] 
\documentclass{article}
\usepackage[T1]{fontenc}
\usepackage{gradient-text}
\usepackage[many]{tcolorbox}
\definecolor{backgroundd}{HTML}{232629}
\definecolor{myblue}{HTML}{1B1F23}
\newtcbox{\cooltextbox}[1][myblue]{
tcbox width=auto limited,
fontupper=\Large\sffamily\bfseries,
colback=#1,colframe=#1,boxsep=0pt,
size=small,arc=2mm,boxrule=10.4pt,
top=2mm,bottom=2mm,right=2mm,left=2mm,
}
\newcommand{\cooltext}[1]{%
\cooltextbox{%
\gradientRGB{#1}{72,107,234}{60,214,112}%
}}

\begin{document}
Normal text
\cooltext{We wanted to make sure everyone was recognized for their 
contributions to this incredible human achievement.}
Normal text
\end{document}
\end{lstlisting}
\end{minipage}
\end{tabular}
%#################### 4.19 ####################
\subsection{\hll{One side dashed border}}
\begin{tabular}{c | c}
\begin{minipage}[m]{0.4\textwidth}
This is \myboxfournine{the box} and what comes next.

This is \myboxfournine[on line]{the same box but \texttt{on line}} and what comes
next.

\bigskip \noindent
Variant based on \verb|\newtcolorbox|:
\begin{myleftlineboxx}
\lipsum[1]
\end{myleftlineboxx}
\end{minipage}
&
\begin{minipage}[m]{0.55\textwidth}
\renewcommand\textminus{\mbox{-}}%<<<<<<<<<<<
\begin{lstlisting}[numberstyle=\zebra{green!15}{yellow!15},numbers=left,basicstyle=\ttfamily\footnotesize] 
\documentclass{article}
\usepackage{kantlipsum}         % for sample text
\usepackage{tcolorbox}
\tcbuselibrary{skins}
\newtcbox{\mybox}[1][]{
    enhanced, frame hidden, borderline west = {0.5pt}{0pt}{red,dashed}, #1
}
\newtcolorbox{myleftlinebox}[1][]{
    enhanced, frame hidden, borderline west = {0.5pt}{0pt}{red,dashed}, #1
}
\begin{document}
This is \mybox{the box} and what comes next.

\bigskip
This is \mybox[on line]{the same box but \texttt{on line}} and what comes
next.

\bigskip \noindent
Variant based on \verb|\newtcolorbox|:
\begin{myleftlinebox}
  \kant[1]
\end{myleftlinebox}
\end{document}
\end{lstlisting}
\end{minipage}
\end{tabular}

%#################### 4.20 ####################
\subsection{\hll{Multilevel Colored Boxes (multi)\xmyboxg{$\backslash$tcblower}}}
\begin{tabular}{c | c}
\begin{minipage}[m]{0.4\textwidth}
\begin{tcolorbox}[colback=red!5,colframe=red!75!black,title=My nice heading]
This is another \textbf{tcolorbox}.

\tcblower

Here, you see the lower part of the box.

\DrawLine

and some more
\end{tcolorbox}
\end{minipage}
&
\begin{minipage}[m]{0.55\textwidth}
\renewcommand\textminus{\mbox{-}}%<<<<<<<<<<<
\begin{lstlisting}[numberstyle=\zebra{green!15}{yellow!15},numbers=left,basicstyle=\ttfamily\footnotesize] 
\documentclass{report}
\usepackage{tikz,tcolorbox}
\makeatletter
\newcommand{\DrawLine}{%
\begin{tikzpicture}
\path[use as bounding box] (0,0) -- (\linewidth,0);
\draw[color=red!75!black,dashed,dash phase=2pt]
(0-\kvtcb@leftlower-\kvtcb@boxsep,0)--
(\linewidth+\kvtcb@rightlower+\kvtcb@boxsep,0);
\end{tikzpicture}%
}
\makeatother

\begin{document}
\begin{tcolorbox}[colback=red!5,colframe=red!75!black,title=My nice heading]
This is another \textbf{tcolorbox}.

\tcblower

Here, you see the lower part of the box.

\DrawLine

and some more
\end{tcolorbox}
\end{document}
\end{lstlisting}
\end{minipage}
\end{tabular}
%#################### 4.21 ####################
%#################### 4.22 ####################
%#################### 4.23 ####################
%#################### 4.24 ####################
%#################### 4.25 ####################
%#################### 4.26 ####################
%#################### 4.27 ####################
%#################### 4.28 ####################











 
 