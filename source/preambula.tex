\usepackage{scrextend}
\usepackage[T1]{fontenc}
\usepackage[utf8]{inputenc}
\usepackage[russian,english]{babel}
\usepackage{amsmath,amssymb}
\usepackage{array,tabularx}
\usepackage{graphicx}
\graphicspath{{C:/Users/user/Desktop/eBook/images}}
\usepackage{moreverb}
\usepackage[table]{xcolor}
\usepackage{color}
\usepackage{lipsum}
%\usepackage{hyperref}
\usepackage{subcaption}
\usepackage[colorlinks=true,linkcolor=black,anchorcolor=black,citecolor=black,filecolor=black,menucolor=black,runcolor=black,urlcolor=black]{hyperref}

%----------------------------- for spacing
\linespread{1.1}
%-----------------------------
\usepackage[most]{tcolorbox}
%\usepackage[many]{tcolorbox}
\newtcbox{\mylib}{enhanced,nobeforeafter,tcbox raise base,boxrule=0.4pt,top=0mm,bottom=0mm,
right=0mm,left=4mm,arc=1pt,boxsep=2pt,before upper={\vphantom{dlg}},
colframe=green!50!black,coltext=green!25!black,colback=green!10!white,
overlay={\begin{tcbclipinterior}\fill[green!75!blue!50!white] (frame.south west)
rectangle node[text=white,font=\sffamily\bfseries\tiny,rotate=90] {TYP} ([xshift=4mm]frame.north west);\end{tcbclipinterior}}}

%----------------------------- for cover
\newtcbox{\forcover}{enhanced,nobeforeafter,tcbox raise base,boxrule=0.4pt,top=0mm,bottom=0mm,
right=0mm,left=4mm,arc=1pt,boxsep=2pt,before upper={\vphantom{dlg}}, colframe=red!50!black,coltext=red!25!black,colback=red!10!white,
overlay={\begin{tcbclipinterior}\fill[red!75!blue!50!white] (frame.south west)
rectangle node[text=white,font=\sffamily\bfseries\tiny,rotate=90] {I N} ([xshift=4mm]frame.north west);\end{tcbclipinterior}}}

%------------------  
\newcolumntype{Y}{>{\centering\arraybackslash}X}
%------------------  
\usepackage{lstautogobble}  % Fix relative indenting
\usepackage{zi4}            % Nice font
	\definecolor{bluekeywords}{rgb}{0.13, 0.13, 1}
	\definecolor{greencomments}{rgb}{0, 0.5, 0}
	\definecolor{redstrings}{rgb}{0.9, 0, 0}
	\definecolor{graynumbers}{rgb}{0.5, 0.5, 0.5}

	\usepackage{listings}
	\lstset{
  basicstyle=\ttfamily,
  columns=fullflexible,
  frame=single,
  breaklines=true,
  %postbreak=\mbox{\textcolor{red}{$\hookrightarrow$}\space},
  }
%------------------
  \newcommand\realnumberstyle[1]{}

\makeatletter
\newcommand{\zebra}[3]{%
    {\realnumberstyle{#3}}%
    \begingroup
    \lst@basicstyle
    \ifodd\value{lstnumber}%
        \color{#1}%
    \else
        \color{#2}%
    \fi
        \rlap{\hspace*{\lst@numbersep}%
        \color@block{\linewidth}{\ht\strutbox}{\dp\strutbox}%
        }%
    \endgroup
}
\makeatother
%------------------
\usepackage[object=vectorian]{pgfornament} %%  http://altermundus.com/pages/tkz/ornament/index.html
		\usepackage{tikz}
		\newcommand{\sectionlinetwo}[2]{%
		\nointerlineskip \vspace{.5\baselineskip}\hspace{\fill}
		{\color{#1}
		\resizebox{0.5\linewidth}{2ex}
		{{{\begin{tikzpicture}
		\node  (C) at (0,0) {};\node (D) at (9,0) {};
		\path (C) to [ornament=#2] (D);
		\end{tikzpicture}}}}}%
		\hspace{\fill}
		\par\nointerlineskip \vspace{.5\baselineskip}}

  %\newcommand{\img}[4]{\center{\includegraphics[width=#1\linewidth]{#2}}\captionof{figure}{#3}\label{#4}}
\newcommand{\img}[2]{\includegraphics[width=#1\linewidth]{#2}}


%-----------------------------
%\usepackage{tcolorbox}
\tcbuselibrary{skins,breakable}
\usetikzlibrary{shadings,shadows}%preambule
%-----------------------------
\newtcbox{\mybox}[1][red]{on line,
arc=0pt,outer arc=0pt,colback=#1!10!white,colframe=#1!50!black,
boxsep=0pt,left=1pt,right=1pt,top=2pt,bottom=2pt,
boxrule=0pt,bottomrule=1pt,toprule=1pt}

\newtcbox{\xmybox}[1][red]{on line,
arc=7pt,colback=#1!10!white,colframe=#1!50!black,
before upper={\rule[-3pt]{0pt}{10pt}},boxrule=1pt,
boxsep=0pt,left=6pt,right=6pt,top=2pt,bottom=2pt}
%-----------------------------
\usepackage{tikz}
\usetikzlibrary{shapes.callouts}
%-----------------------------
\definecolor{amber}{rgb}{1.0, 0.75, 0.0}
\definecolor{babyblue}{rgb}{0.54, 0.81, 0.94}
%-----------------------------

	\usepackage[framemethod=TikZ]{mdframed}
	\usetikzlibrary{calc}
	\makeatletter
	\newlength{\mylength}
	\xdef\CircleFactor{1.1}
	\setlength\mylength{\dimexpr\f@size pt}
	\newsavebox{\myboxx}
	\newcommand*\circled[2][draw=blue]{\savebox\myboxx{\vbox{\vphantom{WL1/}#1}}\setlength\mylength{\dimexpr\CircleFactor\dimexpr\ht\myboxx+\dp\myboxx\relax\relax}\tikzset{mystyle/.style={circle,#1,minimum height={\mylength}}}
	\tikz[baseline=(char.base)]
	\node[mystyle] (char) {#2};}
	\makeatother
%-----------------------------
\usepackage{verbatim}

\usetikzlibrary{arrows,shapes,backgrounds}
%-----------------------------
\usepackage{multicol}
%-----------------------------
%\usepackage[most]{tcolorbox}
	\definecolor{orang}{RGB}{255,155,0}
	\newtcolorbox[auto counter,number within=section]{caja}[1][]{
	enhanced jigsaw,colback=white,colframe=orang,coltitle=orang,
	fonttitle=\bfseries\sffamily,
	sharp corners,
	detach title,
	leftrule=10mm,
	% What you need %%%%%%%%%%%%
	underlay unbroken and first={\node[below,text=black,anchor=east]
	at ([xshift=-5.5pt]interior.base west) {\Huge  \textbf{!}};},
	%%%%%%%%%%%%%%%%%%%%%%%%
	breakable,pad at break=1mm,
	#1,
	code={\ifdefempty{\tcbtitletext}{}{\tcbset{before upper={\tcbtitle\par\medskip}}}},}
%-----------------------------
\newcommand{\enum}[2]{\begin{tikzpicture}
\node (0,0) {\begin{minipage}[m]{0.90\textwidth}
#1
\end{minipage} };
\node [opacity=0.05] (0,0) {\scalebox{8.0}{\textcolor{red}{#2}}};
\end{tikzpicture}}
%-----------------------------

\usepackage{array, makecell}
 \setcellgapes{5pt}
%-----------------------------
\usepackage{pgf-pie}
%-----------------------------
 
\usepackage{nicematrix}
\NiceMatrixOptions{cell-space-top-limit=5pt,cell-space-bottom-limit=5pt}
%-----------------------------
\def\mywd{35pt}
%-----------------------------
\usepackage{booktabs}
\usepackage{multirow}
\usepackage{pdflscape}
\usepackage{longtable, array}
%-----------------------------8.4
\newcommand{\hly}[2]{\colorbox{#1!80}{\parbox{\textwidth}{#2}}}

%-----------------------------4.2
\newtcolorbox{mycolorbox}[1]{
    enhanced,
    breakable,
    title=#1,
    colback=white,
    colbacktitle=green!20!white,
    coltitle=black,
    fonttitle=\bfseries,
    boxrule=.5pt,
    arc=0pt,
    outer arc=0pt,
    colframe=yellow!80!orange,
    borderline west={2pt}{0pt}{red}   }
%-----------------------------4.5
\usepackage{lettrine}
%-----------------------------
\usepackage{coffee4}

%-----------------------------9.2
\usetikzlibrary{fadings, shadings}
\newcounter{fadcnt}\setcounter{fadcnt}{0}
\newcommand\fadingtext[3][]{%
\stepcounter{fadcnt}
  \begin{tikzfadingfrompicture}[name=fading letter\thefadcnt]
    \node[text=transparent!0,inner xsep=0pt,outer xsep=0pt,#1] {#3};
  \end{tikzfadingfrompicture}%
  \begin{tikzpicture}[baseline=(textnode.base)]
    \node[inner sep=0pt,outer sep=0pt,#1](textnode){\phantom{#3}}; 
    \shade[path fading=fading letter\thefadcnt,#2,fit fading=false]
    (textnode.south west) rectangle (textnode.north east);% 
  \end{tikzpicture}% 
}
\usetikzlibrary{calc}
\newbox\shbox
\tikzset{%
  path picture shading/.style={%
  path picture={%
%
\pgfpointdiff{\pgfpointanchor{path picture bounding box}{south west}}%
  {\pgfpointanchor{path picture bounding box}{north east}}%
\pgfgetlastxy\pathwidth\pathheight%
\pgfinterruptpicture%
   \global\setbox\shbox=\hbox{\pgfuseshading{#1}}%
 \endpgfinterruptpicture%
\pgftransformshift{\pgfpointanchor{path picture bounding box}{center}}%
\pgftransformxscale{\pathwidth/(\wd\shbox)}%
\pgftransformyscale{\pathheight/(\ht\shbox)}% \dp will (should) be 0pt
\pgftext{\box\shbox}%
%
    }
  }
}
\pgfdeclarehorizontalshading{rainbow}{10bp}{color(0bp)=(violet);
            color(1.6667bp)=(blue);
            color(3.3333bp)=(cyan);
            color(5bp)=(green);
            color(6.6667bp)=(yellow);
            color(8.3333bp)=(orange);
            color(10bp)=(red)}
%-----------------------------6.3
\definecolor{amethyst}{rgb}{0.6, 0.4, 0.8}
\definecolor{applegreen}{rgb}{0.55, 0.71, 0.0}
\definecolor{arylideyellow}{rgb}{0.91, 0.84, 0.42}
\definecolor{asparagus}{rgb}{0.53, 0.66, 0.42}
\definecolor{atomictangerine}{rgb}{1.0, 0.6, 0.4}
\definecolor{bananayellow}{rgb}{1.0, 0.88, 0.21}
\definecolor{brightgreen}{rgb}{0.4, 1.0, 0.0}
\definecolor{cambridgeblue}{rgb}{0.64, 0.76, 0.68}
\definecolor{capri}{rgb}{0.0, 0.75, 1.0}
\definecolor{carnationpink}{rgb}{1.0, 0.65, 0.79}
 
% from https://tex.stackexchange.com/a/167024/121799
\newcommand{\ClaudioList}{red,applegreen,amethyst,carnationpink,blue!50!cyan,arylideyellow,asparagus,atomictangerine,bananayellow,brightgreen,cambridgeblue,capri}
\newcommand{\SebastianoItem}[1]{\foreach \X[count=\Y] in \ClaudioList
{\ifnum\Y=#1\relax
\xdef\SebastianoColor{\X}
\fi
}
\tikz[baseline=(SebastianoItem.base),remember
picture]{%
\node[fill=\SebastianoColor,inner sep=4pt,font=\sffamily,fill opacity=0.5] (SebastianoItem){#1)};}
}
\newcommand{\SebastianoHighlight}{\tikz[overlay,remember picture]{%
\fill[\SebastianoColor,fill opacity=0.5] ([yshift=4pt,xshift=-\pgflinewidth]SebastianoItem.east) -- ++(4pt,-4pt)
-- ++(-4pt,-4pt) -- cycle;
}}
%-----------------------------2.2
\usepackage{blindtext}
\newcommand*\justify{%
  \fontdimen2\font=0.4em% interword space
  \fontdimen3\font=0.2em% interword stretch
  \fontdimen4\font=0.1em% interword shrink
  \fontdimen7\font=0.1em% extra space
  \hyphenchar\font=`\-% allowing hyphenation
}
 
%-----------------------------9.3
\usepackage{xparse}
\usepackage{fancypar}
\usetikzlibrary{calc,shadows}
 
\NewDocumentCommand\StickyNoteP{O{4cm}mO{4cm}}{%
\begin{tikzpicture}[thick,scale=0.5]
\node[
drop shadow={
  shadow xshift=3pt,
},
inner xsep=0pt,
xslant=-0.1,
yslant=0.1,
inner ysep=0pt,
text depth=\the\dimexpr#1+2.5ex\relax
] {\parbox[t][#1][c]{#3}{#2}};
\end{tikzpicture}%
}
%-----------------------------2.2
\usepackage{ulem}
\usepackage{cancel}
%-----------------------------4.14
\usetikzlibrary{chains,patterns,shadows,fit,backgrounds}

\makeatletter
\tikzset{% customization of pattern
         % based on <m.wibrow@gm...> - 2013-03-24 07:20: 
        hatch distance/.store in=\hatchdistance,
        hatch distance=5pt,
        hatch thickness/.store in=\hatchthickness,
        hatch thickness=5pt
        }
\pgfdeclarepatternformonly[\hatchdistance,\hatchthickness]{north east hatch}% name
    {\pgfqpoint{-1pt}{-1pt}}% below left
    {\pgfqpoint{\hatchdistance}{\hatchdistance}}% above right
    {\pgfpoint{\hatchdistance-1pt}{\hatchdistance-1pt}}%
    {
        \pgfsetcolor{\tikz@pattern@color}
        \pgfsetlinewidth{\hatchthickness}
        \pgfpathmoveto{\pgfqpoint{0pt}{0pt}}
        \pgfpathlineto{\pgfqpoint{\hatchdistance}{\hatchdistance}}
        \pgfusepath{stroke}
    }
\makeatother
%-----------------------------2.2
\usepackage{roboto}
\usepackage[outline]{contour}
 
%\roboto\huge\contourlength{.15em}
%\contour{gray}{boxed}
%-----------------------------9.4
\usepackage[pages=some]{background}% change "some" to "all" to see WM on all pages 
\backgroundsetup{color=green,opacity=0.3,scale=10, contents={A n M n V}}
%-----------------------------2.4
\usepackage{amsmath,soul}
\usepackage{soulpos}
\ulposdef{\ulnumaux}{%
$\underset{\saveulnum}{\rule[-.7ex]{\ulwidth}{.4pt}}$}
\newcommand{\ulnum}[2]{%
\def\saveulnum{#1}%
\ulnumaux{#2}}
%-----------------------------7.4

	\definecolor{white}{RGB}{255,255,255}
	\definecolor{gray}{HTML}{4D4D4D}
	\definecolor{maingray}{HTML}{B9B9B9}

	\newcommand\skills[1]{ 
	    \begin{tikzpicture}
	        \foreach [count=\i] \x/\y in {#1}{
	            \draw[fill=maingray,maingray] (0,\i) rectangle (6,\i+0.4);
	            \draw[fill=white,gray](0,\i) rectangle (\y,\i+0.4);
	            \node[above right] at (0,\i+0.4) {\x};
	        }
	    \end{tikzpicture}
	}
 
%-----------------------------4.15
\usepackage{emerald}

\usetikzlibrary{decorations.pathmorphing}
\usetikzlibrary{shadows}

\tikzset{decoration={random steps,segment length=2mm,amplitude=0.6pt}}

\newtcbtheorem{mytheo}{Theorem}{
  coltitle=green!80!black,
  colback=lightgray!20,
  colbacktitle=lightgray!20,
  fonttitle=\bfseries\ECFAugie,
  enhanced,
  attach boxed title to top left={yshift=-0.18cm,xshift=-0.5mm},
  boxed title style={
    tikz={rotate=4,transform shape},
    frame code={
      \draw[decorate,fill=lightgray!20] (frame.south west) rectangle (frame.north east);
    }
  },
  frame code={
    \draw[decorate,fill=lightgray!20,drop shadow] (frame.north east) rectangle (frame.south west);
  },
}{th}

%-----------------------------9.7
\usepackage{qrcode}

%----------------------------- 2.5 & 6.6
\usepackage{pifont}
%-----------------------------9.7
%-----------------------------8.5
\usetikzlibrary{tikzmark}
%-----------------------------7.8
\usetikzlibrary{trees}
%-----------------------------7.9
\usetikzlibrary{mindmap}
\usetikzlibrary[mindmap]
%-----------------------------2.6
\newcommand{\invtext}[1]{\tikz[baseline]{\node[rectangle,rounded corners=0.5mm,text=white,fill=black!65,inner sep=2pt,anchor=base]  {#1}}}

\newtcbox{\xmyboxg}[1][gray]{on line, arc=4pt,colback=#1!30!white,colframe=white, before upper={\rule[-3pt]{0pt}{10pt}},boxrule=1pt, boxsep=0pt,left=6pt,right=6pt,top=2pt,bottom=2pt}
%-----------------------------8.6
\tikzstyle{buttonstyle} = [rectangle, fill = black!30, draw = black!80, drop shadow, font={\sffamily\bfseries}, text=white]
\newcommand*{\button}[1]{\tikz{\node[buttonstyle] {#1};}}
%----------------------------- for colorful TOC Dec 24 2022
\usepackage{tocloft}
\newcounter{seccntr}
\setcounter{seccntr}{-1}

\newcommand*{\hnode}[1]{%
    \tikz[remember picture] \node[minimum size=0pt,inner sep=0pt,outer sep=4.5pt] (#1) {};}
% create a node at the beginning of the section entry
\renewcommand{\cftsecfont}{\hnode{P1}\bfseries\Large
\stepcounter{seccntr}%
\ifcase\value{seccntr}%
\tikz[remember picture,overlay] \draw (P1.north west)  [line width={17pt}, red,opacity=0.3] -- ++($(\textwidth,0) + (1ex,0)$);
%---- 0 --
\or\tikz[remember picture,overlay] \draw (P1.north west)  [line width={17pt}, green,opacity=0.4] -- ++($(\textwidth,0) + (1ex,0)$);%---- 1 --
\or\tikz[remember picture,overlay]  \draw (P1.north west)  [line width={17pt}, yellow,opacity=1] -- ++($(\textwidth,0) + (1ex,0)$);%---- 2 --
\or\tikz[remember picture,overlay]  \draw (P1.north west)  [line width={17pt}, blue,opacity=0.6] -- ++($(\textwidth,0) + (1ex,0)$);%---- 3 --
\or\tikz[remember picture,overlay]  \draw (P1.north west)  [line width={17pt}, orange,opacity=0.7] -- ++($(\textwidth,0) + (1ex,0)$);
%---- 4 --
\or\tikz[remember picture,overlay]  \draw (P1.north west)  [line width={17pt}, pink,opacity=0.7] -- ++($(\textwidth,0) + (1ex,0)$);
%---- 5 --
\or\tikz[remember picture,overlay]  \draw (P1.north west)  [line width={17pt}, lime,opacity=0.7] -- ++($(\textwidth,0) + (1ex,0)$);
%---- 6 --
\or\tikz[remember picture,overlay]  \draw (P1.north west)  [line width={17pt}, purple,opacity=0.7] -- ++($(\textwidth,0) + (1ex,0)$);
%---- 7 --
\or\tikz[remember picture,overlay]  \draw (P1.north west)  [line width={17pt}, teal,opacity=0.7] -- ++($(\textwidth,0) + (1ex,0)$);
%---- 8 --
\or\tikz[remember picture,overlay]  \draw (P1.north west)  [line width={17pt}, violet,opacity=0.7] -- ++($(\textwidth,0) + (1ex,0)$);
%---- 9 --
\or\tikz[remember picture,overlay]  \draw (P1.north west)  [line width={17pt}, olive,opacity=0.7] -- ++($(\textwidth,0) + (1ex,0)$);
%---- 10 --
\or\tikz[remember picture,overlay]  \draw (P1.north west)  [line width={17pt}, magenta,opacity=0.7] -- ++($(\textwidth,0) + (1ex,0)$);
%-- default
\else\tikz[remember picture,overlay] \draw (P1.north west)  [line width={17pt}, gray,opacity=0.8] -- ++($(\textwidth,0) + (1ex,0)$);%-- default
\fi  %
}
\renewcommand{\cftsecpagefont}{\bfseries}
%----------------------------- Dec 25 2022 for \subsection highligtning
\newtcbox{\hll}[1][yellow]{on line, arc=7pt,colback=#1!10!white,colframe=#1!50!black, before upper={\rule[-3pt]{0pt}{10pt}},boxrule=1pt, boxsep=0pt,left=6pt, right=6pt,top=2pt,bottom=2pt}
%-----------------------------1.9
\usepackage[most]{tcolorbox}
%----------------------------- latex code inside latex code
 

%-----------------------------7.10

\newif\ifsimplegantttickpositionbelow
\tikzset{
    pics/simple gantt/.style={
        code={
            \ifsimplegantttickpositionbelow
                \path[/tikz/simple gantt/tick] (0,0) -- 
                    ++(0,{-1*\pgfkeysvalueof{/tikz/simple gantt/tick length}}) 
                    node[/tikz/simple gantt/tick label] {\pgfmathprintnumber{0}};
            \else
                \path[/tikz/simple gantt/tick] (0,\pgfkeysvalueof{/tikz/simple gantt/height}) -- 
                    ++(0,{\pgfkeysvalueof{/tikz/simple gantt/tick length}}) 
                    node[/tikz/simple gantt/tick label] {\pgfmathprintnumber{0}};
            \fi
            \foreach \n/\x [count=\i, remember=\x as \lastx (initially 0)] in {#1} {
                \ifsimplegantttickpositionbelow
                    \path[/tikz/simple gantt/tick] ({\x*\pgfkeysvalueof{/tikz/simple gantt/width unit}},0) -- 
                        ++(0,{-1*\pgfkeysvalueof{/tikz/simple gantt/tick length}}) 
                        node[/tikz/simple gantt/tick label] {\pgfmathprintnumber{\x}};
                \else
                    \path[/tikz/simple gantt/tick] ({\x*\pgfkeysvalueof{/tikz/simple gantt/width unit}},\pgfkeysvalueof{/tikz/simple gantt/height}) -- 
                        ++(0,{\pgfkeysvalueof{/tikz/simple gantt/tick length}}) 
                        node[/tikz/simple gantt/tick label] {\pgfmathprintnumber{\x}};
                \fi
                \pgfmathparse{int(mod(\i - 1, \pgfkeysvalueof{/tikz/simple gantt/color cycle length}) + 1)} 
                \global\pgfkeyslet{/tikz/simple gantt/color cycle step}{\pgfmathresult}
                \path[
                    /tikz/simple gantt/box, 
                    fill={simple gantt color \pgfkeysvalueof{/tikz/simple gantt/color cycle step}},
                ]
                    ({\lastx*\pgfkeysvalueof{/tikz/simple gantt/width unit}},0) rectangle 
                    ({\x*\pgfkeysvalueof{/tikz/simple gantt/width unit}},\pgfkeysvalueof{/tikz/simple gantt/height})
                    \pgfextra{\pgfmathparse{\x - \lastx}}
                    \ifdim\pgfmathresult pt < \pgfkeysvalueof{/tikz/simple gantt/label as pin if value below} pt\relax
                        node[/tikz/simple gantt/label, pin={[/tikz/simple gantt/label pin]\pgfkeysvalueof{/tikz/simple gantt/label pin angle}:\n}] {}
                    \else
                        node[/tikz/simple gantt/label] {\n}
                    \fi ;
            }
        }
    },
    simple gantt/color cycle length/.initial={0},
    simple gantt/color cycle step/.initial={1},
    simple gantt/color cycle/.code={
        \foreach \c [count=\i] in {#1} {
            \xglobal\colorlet{simple gantt color \i}{\c}
            \global\pgfkeyslet{/tikz/simple gantt/color cycle length}{\i}
        }
    },
    simple gantt/height/.initial={1cm},
    simple gantt/width unit/.initial={1cm},
    simple gantt/box/.style={},
    simple gantt/label/.style={pos=0.5},
    simple gantt/label pin/.style={above, pin edge={black, thin}, pin distance=0.5cm},
    simple gantt/label pin angle/.initial={90},
    simple gantt/label as pin if value below/.initial={1.5},
    simple gantt/tick/.style={draw},
    simple gantt/tick label/.style={below},
    simple gantt/tick position/.is choice,
    simple gantt/tick position/above/.code={\simplegantttickpositionbelowfalse},
    simple gantt/tick position/below/.code={\simplegantttickpositionbelowtrue},
    simple gantt/tick position={below},
    simple gantt/tick length/.initial={5pt},
    simple gantt/color cycle={blue!50, red!50, green!50},
}
%-----------------------------3.7
\usepackage{minted}
\tcbuselibrary{listings, minted, skins}
\tcbset{listing engine=minted}

\newtcblisting{javalst}{listing only, minted language=java, minted style=paraiso-dark,
colback=bg, enhanced, frame hidden, minted options={fontfamily=fdm, 
fontsize=\scriptsize, tabsize=2, breaklines, autogobble}}
\newtcblisting{javalstt}{listing only, minted language=java, minted style=emacs,
colback=white, enhanced, frame hidden, minted options={fontfamily=fdm, bgcolor=white,
fontsize=\scriptsize, tabsize=2, breaklines, autogobble}}
\newtcblisting{javalsttt}{listing only, minted language=java, minted style=vim,
colback=bg, enhanced, frame hidden, minted options={fontfamily=fdm, 
fontsize=\scriptsize, tabsize=2, breaklines, autogobble}}
\newtcblisting{javalstttt}{listing only, minted language=java, minted style=perldoc,
colback=white, enhanced,  minted options={fontfamily=fdm, bgcolor=white,
fontsize=\scriptsize, tabsize=2, breaklines, autogobble}}
\renewcommand{\theFancyVerbLine}{\textcolor[rgb]{1,1,1}{\scriptsize\arabic{FancyVerbLine}}}
\newtcblisting{python}{
    listing only, 
    minted language=java, 
    minted style=paraiso-dark,
    colback=bg, 
    enhanced, 
    frame hidden,
    minted options={ 
        tabsize=2, 
        breaklines, 
        autogobble,
        linenos,
        numbersep=5pt,
        fontsize=\scriptsize,
    },
    overlay={\begin{tcbclipinterior}\fill[bg](frame.south west)rectangle([xshift=5mm]frame.north west);\end{tcbclipinterior}}
}

\definecolor{inline}{RGB}{187,57,82}
\definecolor{bg}{RGB}{22,43,58}
\setminted[java]{bgcolor=bg, fontfamily=fdm, fontsize=\footnotesize}
%-----------------------------4ю17
\usepackage[export]{adjustbox}
%-----------------------------9.11
\usepackage{lmodern}
\usepackage{tikz}
\usetikzlibrary{calc}
\newcommand{\stripskip}{5}
\newcommand{\stripwidth}{3}

%-----------------------------1.11
\usepackage{tabularray}
%-----------------------------1.8
\usepackage{mathtools}
%-----------------------------7.12
\usepackage{circuitikz}
\ctikzset{bipoles/thickness =1}

\usepackage{needspace}
%-----------------------------4.18
\usepackage{gradient-text}
 
\definecolor{myblue}{HTML}{1B1F23}
\newtcbox{\cooltextbox}[1][myblue]{
  tcbox width=auto limited,
  fontupper=\Large\sffamily\bfseries,
  colback=#1,
  colframe=#1,
  boxsep=0pt,
  size=small,
  arc=2mm,
  boxrule=10.4pt,
  top=2mm,
  bottom=2mm,
  right=2mm,
  left=2mm,
}
\newcommand{\cooltext}[1]{%
  \cooltextbox{%
    \gradientRGB{#1}{72,107,234}{60,214,112}%
  }%
}
%-----------------------------4.19
\newtcbox{\myboxfournine}[1][]{
    enhanced, frame hidden, borderline west = {0.5pt}{0pt}{red,dashed}, #1
}
\newtcolorbox{myleftlineboxx}[1][]{
    enhanced, frame hidden, borderline west = {0.5pt}{0pt}{red,dashed}, #1
}
%-----------------------------

\usepackage{afterpage}
\usepackage{anyfontsize}

%\definecolor{applegreen}{rgb}{0.55, 0.71, 0.0}
\definecolor{darkorange}{rgb}{1.0, 0.55, 0.0}
%-----------------------------7.13
\usetikzlibrary{decorations.text}
\definecolor{mygray}{RGB}{208,208,208}
\definecolor{mymagenta}{RGB}{226,0,116}
\newcommand*{\mytextstyle}{\sffamily\Large\bfseries\color{black!85}}
\newcommand{\arcarrow}[3]{%
% inner radius, middle radius, outer radius, start angle,
% end angle, tip protusion angle, options, text
\pgfmathsetmacro{\rin}{1.7}
\pgfmathsetmacro{\rmid}{2.2}
\pgfmathsetmacro{\rout}{2.7}
\pgfmathsetmacro{\astart}{#1}
\pgfmathsetmacro{\aend}{#2}
\pgfmathsetmacro{\atip}{5}
\fill[mygray, very thick] (\astart+\atip:\rin)
arc (\astart+\atip:\aend:\rin)
-- (\aend-\atip:\rmid)
-- (\aend:\rout)   arc (\aend:\astart+\atip:\rout)
-- (\astart:\rmid) -- cycle;
\path[
decoration = {
text along path,
text = {|\mytextstyle|#3},
text align = {align = center},
raise = -1.0ex
},
decorate
](\astart+\atip:\rmid) arc (\astart+\atip:\aend+\atip:\rmid);
}

%-----------------------------4.20
\makeatletter
\newcommand{\DrawLine}{%
\begin{tikzpicture}
\path[use as bounding box] (0,0) -- (\linewidth,0);
\draw[color=red!75!black,dashed,dash phase=2pt]
(0-\kvtcb@leftlower-\kvtcb@boxsep,0)--
(\linewidth+\kvtcb@rightlower+\kvtcb@boxsep,0);
\end{tikzpicture}%
}
\makeatother

%-----------------------------9.12
\usepackage{shapepar}

%----------------------------- 9.14
\usepackage{tikzlings}

%-----------------------------



%-----------------------------

%-----------------------------

%-----------------------------

%-----------------------------

%-----------------------------

%-----------------------------



