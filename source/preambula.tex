\usepackage{scrextend}
\usepackage[T1]{fontenc}
\usepackage[utf8]{inputenc}
	\usepackage[russian,english]{babel}
	\usepackage{amsmath,amssymb}
	\usepackage{array,tabularx}
	\usepackage{graphicx}
	\graphicspath{{images/}}
	\usepackage{moreverb}
	%\usepackage[svgnames,dvipsnames,x11names,table]{xcolor}
	 \usepackage[table]{xcolor}
	\usepackage{color}
	\usepackage{lipsum}
	\usepackage{hyperref}
	\usepackage{subcaption}
	\hypersetup{%
		colorlinks=true,
		linkbordercolor= red,
		urlbordercolor=green,
    }%for hyperrefs
%------------------ 
 %-----------------------------5.4
%\renewcommand\thesubfigure{{\itshape\alph{subfigure}}}
 
%-----------------------------

\usepackage[many]{tcolorbox}
	\newtcbox{\mylib}{enhanced,nobeforeafter,tcbox raise base,boxrule=0.4pt,top=0mm,bottom=0mm,
	  right=0mm,left=4mm,arc=1pt,boxsep=2pt,before upper={\vphantom{dlg}},
	  colframe=green!50!black,coltext=green!25!black,colback=green!10!white,
	  overlay={\begin{tcbclipinterior}\fill[green!75!blue!50!white] (frame.south west)
	    rectangle node[text=white,font=\sffamily\bfseries\tiny,rotate=90] {TYP} ([xshift=4mm]frame.north west);\end{tcbclipinterior}}}

%------------------  
\newcolumntype{Y}{>{\centering\arraybackslash}X}
%------------------  
\usepackage{lstautogobble}  % Fix relative indenting
%\usepackage{color}          % Code coloring
\usepackage{zi4}            % Nice font
	\definecolor{bluekeywords}{rgb}{0.13, 0.13, 1}
	\definecolor{greencomments}{rgb}{0, 0.5, 0}
	\definecolor{redstrings}{rgb}{0.9, 0, 0}
	\definecolor{graynumbers}{rgb}{0.5, 0.5, 0.5}

	\usepackage{listings}
	\lstset{
  basicstyle=\ttfamily,
  columns=fullflexible,
  frame=single,
  breaklines=true,
  postbreak=\mbox{\textcolor{red}{$\hookrightarrow$}\space},
  }
%------------------
  \newcommand\realnumberstyle[1]{}

\makeatletter
\newcommand{\zebra}[3]{%
    {\realnumberstyle{#3}}%
    \begingroup
    \lst@basicstyle
    \ifodd\value{lstnumber}%
        \color{#1}%
    \else
        \color{#2}%
    \fi
        \rlap{\hspace*{\lst@numbersep}%
        \color@block{\linewidth}{\ht\strutbox}{\dp\strutbox}%
        }%
    \endgroup
}
\makeatother
%------------------
\usepackage[object=vectorian]{pgfornament} %%  http://altermundus.com/pages/tkz/ornament/index.html
		\usepackage{tikz}
		\newcommand{\sectionlinetwo}[2]{%
		\nointerlineskip \vspace{.5\baselineskip}\hspace{\fill}
		{\color{#1}
		\resizebox{0.5\linewidth}{2ex}
		{{{\begin{tikzpicture}
		\node  (C) at (0,0) {};\node (D) at (9,0) {};
		\path (C) to [ornament=#2] (D);
		\end{tikzpicture}}}}}%
		\hspace{\fill}
		\par\nointerlineskip \vspace{.5\baselineskip}}

  %\newcommand{\img}[4]{\center{\includegraphics[width=#1\linewidth]{#2}}\captionof{figure}{#3}\label{#4}}
\newcommand{\img}[2]{\includegraphics[width=#1\linewidth]{#2}}


%-----------------------------
\usepackage{tcolorbox}
\tcbuselibrary{skins,breakable}
\usetikzlibrary{shadings,shadows}%preambule
%-----------------------------
\newtcbox{\mybox}[1][red]{on line,
arc=0pt,outer arc=0pt,colback=#1!10!white,colframe=#1!50!black,
boxsep=0pt,left=1pt,right=1pt,top=2pt,bottom=2pt,
boxrule=0pt,bottomrule=1pt,toprule=1pt}

\newtcbox{\xmybox}[1][red]{on line,
arc=7pt,colback=#1!10!white,colframe=#1!50!black,
before upper={\rule[-3pt]{0pt}{10pt}},boxrule=1pt,
boxsep=0pt,left=6pt,right=6pt,top=2pt,bottom=2pt}
%-----------------------------
\usepackage{tikz}
\usetikzlibrary{shapes.callouts}
%-----------------------------
\definecolor{amber}{rgb}{1.0, 0.75, 0.0}
\definecolor{babyblue}{rgb}{0.54, 0.81, 0.94}
%-----------------------------

	\usepackage[framemethod=TikZ]{mdframed}
	\usetikzlibrary{calc}
	\makeatletter
	\newlength{\mylength}
	\xdef\CircleFactor{1.1}
	\setlength\mylength{\dimexpr\f@size pt}
	\newsavebox{\myboxx}
	\newcommand*\circled[2][draw=blue]{\savebox\myboxx{\vbox{\vphantom{WL1/}#1}}\setlength\mylength{\dimexpr\CircleFactor\dimexpr\ht\myboxx+\dp\myboxx\relax\relax}\tikzset{mystyle/.style={circle,#1,minimum height={\mylength}}}
	\tikz[baseline=(char.base)]
	\node[mystyle] (char) {#2};}
	\makeatother
%-----------------------------
\usepackage{verbatim}

\usetikzlibrary{arrows,shapes,backgrounds}
%-----------------------------
\usepackage{multicol}
%-----------------------------
%\usepackage[most]{tcolorbox}
	\definecolor{orang}{RGB}{255,155,0}
	\newtcolorbox[auto counter,number within=section]{caja}[1][]{
	enhanced jigsaw,colback=white,colframe=orang,coltitle=orang,
	fonttitle=\bfseries\sffamily,
	sharp corners,
	detach title,
	leftrule=10mm,
	% What you need %%%%%%%%%%%%
	underlay unbroken and first={\node[below,text=black,anchor=east]
	at ([xshift=-5.5pt]interior.base west) {\Huge  \textbf{!}};},
	%%%%%%%%%%%%%%%%%%%%%%%%
	breakable,pad at break=1mm,
	#1,
	code={\ifdefempty{\tcbtitletext}{}{\tcbset{before upper={\tcbtitle\par\medskip}}}},}
%-----------------------------
\newcommand{\enum}[2]{\begin{tikzpicture}
\node (0,0) {\begin{minipage}[m]{0.90\textwidth}
#1
\end{minipage} };
\node [opacity=0.05] (0,0) {\scalebox{8.0}{\textcolor{red}{\emph{#2}}}};
\end{tikzpicture}}
%-----------------------------

\usepackage{array, makecell}
 \setcellgapes{5pt}
%-----------------------------
\usepackage{pgf-pie}
%-----------------------------
 
\usepackage{nicematrix}
\NiceMatrixOptions{cell-space-top-limit=5pt,cell-space-bottom-limit=5pt}
%-----------------------------
\def\mywd{35pt}
%-----------------------------
\usepackage{booktabs}

\usepackage{multirow}
\usepackage{pdflscape}
\usepackage{longtable, array}
%-----------------------------8.4
\newcommand{\hly}[2]{\colorbox{#1!80}{\parbox{\textwidth}{#2}}}

%-----------------------------4.2
\newtcolorbox{mycolorbox}[1]{
    enhanced,
    breakable,
    title=#1,
    colback=white,
    colbacktitle=green!20!white,
    coltitle=black,
    fonttitle=\bfseries,
    boxrule=.5pt,
    arc=0pt,
    outer arc=0pt,
    colframe=yellow!80!orange,
    borderline west={2pt}{0pt}{red}   }
%-----------------------------4.5
\usepackage{lettrine}
%-----------------------------
\usepackage{coffee4}

%-----------------------------9.2
\usetikzlibrary{fadings, shadings}
\newcounter{fadcnt}\setcounter{fadcnt}{0}
\newcommand\fadingtext[3][]{%
\stepcounter{fadcnt}
  \begin{tikzfadingfrompicture}[name=fading letter\thefadcnt]
    \node[text=transparent!0,inner xsep=0pt,outer xsep=0pt,#1] {#3};
  \end{tikzfadingfrompicture}%
  \begin{tikzpicture}[baseline=(textnode.base)]
    \node[inner sep=0pt,outer sep=0pt,#1](textnode){\phantom{#3}}; 
    \shade[path fading=fading letter\thefadcnt,#2,fit fading=false]
    (textnode.south west) rectangle (textnode.north east);% 
  \end{tikzpicture}% 
}
\usetikzlibrary{calc}
\newbox\shbox
\tikzset{%
  path picture shading/.style={%
  path picture={%
%
\pgfpointdiff{\pgfpointanchor{path picture bounding box}{south west}}%
  {\pgfpointanchor{path picture bounding box}{north east}}%
\pgfgetlastxy\pathwidth\pathheight%
\pgfinterruptpicture%
   \global\setbox\shbox=\hbox{\pgfuseshading{#1}}%
 \endpgfinterruptpicture%
\pgftransformshift{\pgfpointanchor{path picture bounding box}{center}}%
\pgftransformxscale{\pathwidth/(\wd\shbox)}%
\pgftransformyscale{\pathheight/(\ht\shbox)}% \dp will (should) be 0pt
\pgftext{\box\shbox}%
%
    }
  }
}
\pgfdeclarehorizontalshading{rainbow}{10bp}{color(0bp)=(violet);
            color(1.6667bp)=(blue);
            color(3.3333bp)=(cyan);
            color(5bp)=(green);
            color(6.6667bp)=(yellow);
            color(8.3333bp)=(orange);
            color(10bp)=(red)}
<<<<<<< HEAD
%-----------------------------6.3
\definecolor{amethyst}{rgb}{0.6, 0.4, 0.8}
\definecolor{applegreen}{rgb}{0.55, 0.71, 0.0}
\definecolor{arylideyellow}{rgb}{0.91, 0.84, 0.42}
\definecolor{asparagus}{rgb}{0.53, 0.66, 0.42}
\definecolor{atomictangerine}{rgb}{1.0, 0.6, 0.4}
\definecolor{bananayellow}{rgb}{1.0, 0.88, 0.21}
\definecolor{brightgreen}{rgb}{0.4, 1.0, 0.0}
\definecolor{cambridgeblue}{rgb}{0.64, 0.76, 0.68}
\definecolor{capri}{rgb}{0.0, 0.75, 1.0}
\definecolor{carnationpink}{rgb}{1.0, 0.65, 0.79}
 
% from https://tex.stackexchange.com/a/167024/121799
\newcommand{\ClaudioList}{red,applegreen,amethyst,carnationpink,blue!50!cyan,arylideyellow,asparagus,atomictangerine,bananayellow,brightgreen,cambridgeblue,capri}
\newcommand{\SebastianoItem}[1]{\foreach \X[count=\Y] in \ClaudioList
{\ifnum\Y=#1\relax
\xdef\SebastianoColor{\X}
\fi
}
\tikz[baseline=(SebastianoItem.base),remember
picture]{%
\node[fill=\SebastianoColor,inner sep=4pt,font=\sffamily,fill opacity=0.5] (SebastianoItem){#1)};}
}
\newcommand{\SebastianoHighlight}{\tikz[overlay,remember picture]{%
\fill[\SebastianoColor,fill opacity=0.5] ([yshift=4pt,xshift=-\pgflinewidth]SebastianoItem.east) -- ++(4pt,-4pt)
-- ++(-4pt,-4pt) -- cycle;
}}
%-----------------------------2.2
\usepackage{blindtext}
\newcommand*\justify{%
  \fontdimen2\font=0.4em% interword space
  \fontdimen3\font=0.2em% interword stretch
  \fontdimen4\font=0.1em% interword shrink
  \fontdimen7\font=0.1em% extra space
  \hyphenchar\font=`\-% allowing hyphenation
}
=======
%-----------------------------9.3
\usepackage{xparse}
\usepackage{fancypar}
\usetikzlibrary{calc,shadows}
 
\NewDocumentCommand\StickyNoteP{O{4cm}mO{4cm}}{%
\begin{tikzpicture}[thick,scale=0.5]
\node[
drop shadow={
  shadow xshift=3pt,
},
inner xsep=0pt,
xslant=-0.1,
yslant=0.1,
inner ysep=0pt,
text depth=\the\dimexpr#1+2.5ex\relax
] {\parbox[t][#1][c]{#3}{#2}};
\end{tikzpicture}%
}
%-----------------------------2.2
\usepackage{ulem}
\usepackage{cancel}
%-----------------------------4.14
\usetikzlibrary{chains,patterns,shadows,fit,backgrounds}

\makeatletter
\tikzset{% customization of pattern
         % based on <m.wibrow@gm...> - 2013-03-24 07:20: 
        hatch distance/.store in=\hatchdistance,
        hatch distance=5pt,
        hatch thickness/.store in=\hatchthickness,
        hatch thickness=5pt
        }
\pgfdeclarepatternformonly[\hatchdistance,\hatchthickness]{north east hatch}% name
    {\pgfqpoint{-1pt}{-1pt}}% below left
    {\pgfqpoint{\hatchdistance}{\hatchdistance}}% above right
    {\pgfpoint{\hatchdistance-1pt}{\hatchdistance-1pt}}%
    {
        \pgfsetcolor{\tikz@pattern@color}
        \pgfsetlinewidth{\hatchthickness}
        \pgfpathmoveto{\pgfqpoint{0pt}{0pt}}
        \pgfpathlineto{\pgfqpoint{\hatchdistance}{\hatchdistance}}
        \pgfusepath{stroke}
    }
\makeatother
%-----------------------------
%-----------------------------

%-----------------------------
%-----------------------------
>>>>>>> 9ecf04c70880d4cc1ae635f1b338876afb11cd90
%-----------------------------

%-----------------------------
%-----------------------------
%-----------------------------

%-----------------------------
%-----------------------------
%-----------------------------

%-----------------------------
%-----------------------------
%-----------------------------
%-----------------------------